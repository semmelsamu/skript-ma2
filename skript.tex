\documentclass{article}

\usepackage[margin=2cm]{geometry}

\usepackage[utf8]{inputenc}
\usepackage[german]{babel}

\usepackage{amssymb}
\usepackage{amsmath}
\usepackage{amsthm}
\usepackage{enumitem}

\setlength{\parindent}{0pt}


\newtheoremstyle{break}
  {0}{3em}%
  {}{}%
  {\bfseries}{}%
  {\newline}%
  {\thmname{#1}\thmnumber{ #2}\thmnote{ (#3)}\newline\indent}%


\theoremstyle{break}

\newtheorem{definition}{Definition}
\newtheorem{korollar}{Korollar}
\newtheorem{satz}{Satz}

\newcommand{\entry}[3]{\begin{#1}[#2]\input{tex/#3}\end{#1}}



\begin{document}

\section{Reelle Zahlen}

% \entry{definition}{Natürliche Zahlen}{natuerliche_zahlen}

\entry{definition}{Kommutativer Körper}{kommutativer_koerper}

\entry{definition}{Gesetze im Körper}{gesetze_im_koerper}

\entry{definition}{Angeordneter Körper}{angeordneter_koerper}

\entry{definition}{Archimedisch angrordneter Körper}{archimedisch_angeordneter_körper}

\entry{korollar}{Gesetze im angeordneten Körper}{gesetze_im_angeordneten_koerper}


\entry{definition}{Diskrete Menge}{diskrete_menge}

\entry{definition}{Fakultät}{fakultaet}


\end{document}