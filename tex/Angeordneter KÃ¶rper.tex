Ein kommutativer Körper heißt \bi{angeordneter Körper}, falls in $K$ eine Menge der \bi{positiven Elemente} (Schreibweise: $a > 0$ steht für "$a$ ist postitiv") ausgezeichnet ist, die folgende Eigenschaften besitzt:

\begin{enumerate}

    \item Für jedes $a \in K$ gilt genau eine der drei Bedingungen:
    $$\text{$a > 0$ oder $a = 0$ oder $-a > 0$}$$

    \item Aus $a > 0$ und $b > 0$ folgen $a+b > 0$ sowie $ab > 0$.

\end{enumerate}