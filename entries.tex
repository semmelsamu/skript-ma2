\section{Reelle Zahlen}

    \subsection{Archimedisch angrordnete Körper}

        \entry{Definition}{Kommutativer Körper}
        
        TODO: $a^-1$ schreibweise $1/a$, ...
        
        \entry{Korollar}{Gesetze im Körper}
        \entry{Definition}{Angeordneter Körper}
        \entry{Definition}{Archimedisch angeordneter Körper}
        
        TODO: Schreibweisen
        
        \entry{Korollar}{Gesetze im angeordneten Körper}
        
        \entry{Satz}{Bernoulli-Ungleichung}
        \entry{Satz}{Potenzen auf Körpern}
        
        \entry{Definition}{Absolutbetrag}
        \entry{Satz}{Rechenregeln von Beträgen}
        
        \entry{Notation}{Summe von Körperelementen}
        \entry{Notation}{Fakultät}
        \entry{Notation}{Binomialkoeffizient}
        \entry{Beispiel}{Binomialkoeffizient}[Binomialkoeffizient - Beispiel]
        \entry{Satz}{Binomischer Lehrsatz}
        
    \subsection{Intervallschachtelung und Vollständigkeit}
    
        TODO: Monoton wachsend, monoton fallend
    
        \entry{Definition}{Intervall}
        \entry{Definition}{Länge}[Länge eines Intervalls]
        \entry{Definition}{Intervallschachtelung}
        \entry{Definition}{Vollständiger Körper}
        \entry{Satz}{Vollständigkeitsaxiom}
        
        \entry{Beispiel}{Bestimmung der Eulerschen Zahl}[Bestimmung der Eulerschen Zahl durch Intervallschachtelung]

        \entry{Satz}{Existenz von Wurzeln}
        \entry{Lemma}{Ungleichung für Wurzeln}
        
    \subsection{Supremuseigenschaft}
    
        \entry{Definition}{Untere und obere Schranke}
        \entry{Definition}{Minimum, Maximum}
        \entry{Beispiel}{Untere Schranke, obere Schranke, Minimum, Maximum}[Untere Schranke, obere Schranke, Minimum, Maximum - Beispiel]
        \entry{Definition}{Infimum, Supremum}
        \entry{Bemerkung}{Unterschied zwischen Supremum und Maximum}
        \entry{Bemerkung}{Kontingenz von Infimum bzw. Supremum}
        \entry{Satz}{Existenz von Infimum bzw. Supremum}
        
        \entry{Lemma}{Rechenregeln bei beschränkten Teilmengen}