\section{Reelle Zahlen}

    \subsection{Archimedisch angrordnete Körper}

        \entry{Definition}{Kommutativer Körper}
        
        TODO: $a^-1$ schreibweise $1/a$, ...
        
        \entry{Korollar}{Gesetze im Körper}
        \entry{Definition}{Angeordneter Körper}
        \entry{Definition}{Archimedisch angeordneter Körper}
        
        TODO: Schreibweisen
        
        \entry{Korollar}{Gesetze im angeordneten Körper}
        
        \entry{Satz}{Bernoulli-Ungleichung}
        \entry{Satz}{Potenzen auf Körpern}
        
        \entry{Definition}{Absolutbetrag}
        \entry{Satz}{Rechenregeln von Beträgen}
        
        \entry{Notation}{Summe von Körperelementen}
        \entry{Notation}{Fakultät}
        \entry{Notation}{Binomialkoeffizient}
        \entry{Beispiel}{Binomialkoeffizient}[Binomialkoeffizient - Beispiel]
        \entry{Satz}{Binomischer Lehrsatz}
        
    \subsection{Intervallschachtelung und Vollständigkeit}
    
        TODO: Monoton wachsend, monoton fallend
    
        \entry{Definition}{Intervall}
        \entry{Definition}{Länge}[Länge eines Intervalls]
        \entry{Definition}{Intervallschachtelung}
        \entry{Definition}{Vollständiger Körper}
        \entry{Satz}{Vollständigkeitsaxiom}
        
        \entry{Beispiel}{Bestimmung der Eulerschen Zahl}[Bestimmung der Eulerschen Zahl durch Intervallschachtelung]

        \entry{Satz}{Existenz von Wurzeln}
        \entry{Lemma}{Ungleichung für Wurzeln}
        
    \subsection{Supremuseigenschaft}
    
        \entry{Definition}{Untere und obere Schranke}
        \entry{Definition}{Minimum, Maximum}
        \entry{Beispiel}{Schranken und Extrema}
        \entry{Definition}{Infimum, Supremum}
        \entry{Bemerkung}{Unterschied zwischen Grenzwert und Extremum}
        \entry{Bemerkung}{Kontingenz von Grenzwerten}
        \entry{Satz}{Existenz von Grenzwerten}
        
        \entry{Lemma}{Rechenregeln bei beschränkten Teilmengen}
        \entry{Satz}{Extremwerte in natürlichen Zahlen}
        \entry{Satz}{Existenz von Zahlen in reellen Intervallen}
        
\section{Der Körper $\mathbb{C}$ der komplexen Zahlen}

    \subsection{Komplexe Zahlen}
    
        \entry{Definition}{Komplexe Zahlen}
        \entry{Definition}{Rechenoperationen auf komplexen Zahlen}
        \entry{Lemma}{Neutrale und inverse Elemente der komplexen Zahlen}
        \entry{Definition}{Körper der komplexen Zahlen}
        \entry{Bemerkung}{Rechenoperationen von reellen Zahlen in der komplexen Ebene}
        \entry{Bemerkung}{Reelle Zahlen als Teilmenge der komplexen Zahlen}
        
    \subsection{Darstellung komplexer Zahlen}
        
        \entry{Definition}{Imaginäre Einheit}
        \entry{Definition}{Normaldarstellung komplexer Zahlen}
        \entry{Definition}{Konjugiert komplexe Zahl}
        \entry{Definition}{Betrag einer komplexen Zahl} 
        \entry{Satz}{Rechenregeln der komplexen Zahlen} 
        \entry{Satz}{Dreiecksungleichung} 
        \entry{Satz}{Fundamentalsatz der Algebra} 
        \entry{Bemerkung}{Geltungsbereich des Fundamentalsatz der Algebra} 
        
        TODO: Beispiel - Dritte Einheitswurzeln
        
        \entry{Definition}{Polardarstellung komplexer Zahlen} 
        \entry{Satz}{Lösung von Wurzeln in komplexen Zahlen} 
        \entry{Satz}{Potenzen komplexer Zahlen in Polardarstellung} 
        \entry{Satz}{Formel von Moivre} 
        
\section{Folgen reeller und komplexer Zahlen}

    \subsection{Folgen und Grenzwerte}
    
        \entry{Definition}{Folge}
        \entry{Definition}{Konvergenz, Grenzwert}
        \entry{Notation}{Grenzwert}[Grenzwert - Notation]
        \entry{Bemerkung}{Eindeutigkeit von Grenzwerten}
        \entry{Satz}{Beschränktheit konvergenter Folgen}
        \entry{Definition}{Monotonie}
        \entry{Satz}{Konvergenz monotoner Folgen}
        \entry{Korollar}{Grenzwerte für Intervallschachtelungen}
        \entry{Satz}{Rechenregeln für Grenzwerte}
        \entry{Notation}{Grenzwerte}[Grenzwerte - Notation]
        \entry{Lemma}{Konvergente Grenzwerte}
        
        \entry{Lemma}{Sandwich-Lemma}
        \entry{Bemerkung}{Wichtige Grenzwerte}
        \entry{Satz}{Grenzwerte rationaler Funktionen}
        
        \entry{Definition}{Asymptotisch gleich}
        
    \subsection{Eine Intervallschachtelung für den natürlichen Logarithmus}
    
        \entry{Satz}{Konvergenz komplexer Zahlenfolgen}