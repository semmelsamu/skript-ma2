\section{Reelle Zahlen}

    \subsection{Archimedisch angrordnete Körper}

        \entry{Definition}{Kommutativer Körper}
        \entry{Notation}{Schreibweisen im kommutativen Körper}
        
        \entry{Korollar}{Gesetze im Körper}
        \entry{Definition}{Angeordneter Körper}
        \entry{Definition}{Archimedisch angeordneter Körper}
        
        \entry{Notation}{Schreibweisen zur Ungleichheit}
        
        \entry{Korollar}{Gesetze im angeordneten Körper}
        
        \entry{Satz}{Bernoulli-Ungleichung}
        \entry{Satz}{Potenzen auf Körpern}
        
        \entry{Definition}{Absolutbetrag}
        \entry{Satz}{Rechenregeln von Beträgen}
        
        \entry{Notation}{Summe von Körperelementen}
        \entry{Notation}{Fakultät}
        \entry{Notation}{Binomialkoeffizient}
        \entry{Beispiel}{Binomialkoeffizient}[Binomialkoeffizient - Beispiel]
        \entry{Satz}{Binomischer Lehrsatz}
        
    \subsection{Intervallschachtelung und Vollständigkeit}
    
        \entry{Definition}{Monotonie}
    
        \entry{Definition}{Intervall}
        \entry{Definition}{Länge}[Länge eines Intervalls]
        \entry{Definition}{Intervallschachtelung}
        \entry{Definition}{Vollständiger Körper}
        \entry{Satz}{Vollständigkeitsaxiom}
        
        \entry{Beispiel}{Bestimmung der Eulerschen Zahl}[Bestimmung der Eulerschen Zahl durch Intervallschachtelung]

        \entry{Satz}{Existenz von Wurzeln}
        \entry{Lemma}{Ungleichung für Wurzeln}
        
        \entry{Definition}{Injektiv, surjektiv, bijektiv}
        
    \subsection{Supremuseigenschaft}
    
        \entry{Definition}{Untere und obere Schranke}
        \entry{Definition}{Minimum, Maximum}
        \entry{Beispiel}{Schranken und Extrema}
        \entry{Definition}{Infimum, Supremum}
        \entry{Bemerkung}{Unterschied zwischen Grenzwert und Extremum}
        \entry{Bemerkung}{Kontingenz von Grenzwerten}
        \entry{Satz}{Existenz von Grenzwerten}
        
        \entry{Lemma}{Rechenregeln bei beschränkten Teilmengen}
        \entry{Satz}{Extremwerte in natürlichen Zahlen}
        \entry{Satz}{Existenz von Zahlen in reellen Intervallen}
        
\section{Der Körper der komplexen Zahlen}

    \subsection{Komplexe Zahlen}
    
        \entry{Definition}{Komplexe Zahlen}
        \entry{Definition}{Rechenoperationen auf komplexen Zahlen}
        \entry{Lemma}{Neutrale und inverse Elemente der komplexen Zahlen}
        \entry{Definition}{Körper der komplexen Zahlen}
        \entry{Bemerkung}{Rechenoperationen von reellen Zahlen in der komplexen Ebene}
        \entry{Bemerkung}{Reelle Zahlen als Teilmenge der komplexen Zahlen}
        
    \subsection{Darstellung komplexer Zahlen}
        
        \entry{Definition}{Imaginäre Einheit}
        \entry{Definition}{Normaldarstellung komplexer Zahlen}
        \entry{Definition}{Konjugiert komplexe Zahl}
        \entry{Definition}{Betrag einer komplexen Zahl} 
        \entry{Satz}{Rechenregeln der komplexen Zahlen} 
        \entry{Satz}{Dreiecksungleichung} 
        \entry{Satz}{Fundamentalsatz der Algebra} 
        \entry{Bemerkung}{Geltungsbereich des Fundamentalsatz der Algebra}
        
        \entry{Definition}{Polardarstellung komplexer Zahlen} 
        % \entry{Satz}{Formeln zur Darstellung komplexer Zahlen in Polarform}
        \entry{Satz}{Lösung von Wurzeln in komplexen Zahlen} 
        \entry{Satz}{Potenzen komplexer Zahlen in Polardarstellung} 
        \entry{Satz}{Formel von Moivre} 
        
\section{Folgen reeller und komplexer Zahlen}

    \subsection{Folgen und Grenzwerte}
    
        \entry{Definition}{Folge}
        \entry{Bemerkung}{Unterschied zwischen Folge und Menge der Folgenglieder}
        \entry{Definition}{Konvergenz, Grenzwert}
        \entry{Notation}{Grenzwert}[Grenzwert - Notation]
        \entry{Bemerkung}{Eindeutigkeit von Grenzwerten}
        \entry{Satz}{Beschränktheit konvergenter Folgen}
        
        \entry{Satz}{Algorithmus zur Bestimmung der Quadratwurzel}
        
    \subsection{Monotonie}
        
        \entry{Definition}{Monoton wachsend, monoton fallend}[Monotonie]
        \entry{Satz}{Konvergenz monotoner Folgen}
        \entry{Korollar}{Grenzwerte für Intervallschachtelungen}
        \entry{Satz}{Rechenregeln für Grenzwerte}
        \entry{Notation}{Grenzwerte}[Grenzwerte - Notation]
        \entry{Lemma}{Konvergente Grenzwerte}
        
        \entry{Lemma}{Sandwich-Lemma}
        \entry{Bemerkung}{Wichtige Grenzwerte}
        \entry{Satz}{Grenzwerte rationaler Funktionen}
        
        \entry{Definition}{Asymptotisch gleich}
        
    \subsection{Eine Intervallschachtelung für den natürlichen Logarithmus}
    
        \entry{Satz}{Rechenregeln für den natürlichen Logarithmus}
        \entry{Korollar}{Rechenregeln für den natürlichen Logarithmus}[Natürlicher Logarithmus - Korollar]
        \entry{Satz}{Konvergenz komplexer Zahlenfolgen}
        \entry{Korollar}{Konvergenz komplexer Zahlenfolgen}[Konvergenz komplexer Zahlenfolgen - Korollar]
        
\section{Metrische Räume und Cauchy-Folge}

    \subsection{Metrische Räume}
    
        \entry{Definition}{Metrik}
        \entry{Definition}{Norm}
        \entry{Bemerkung}{Metrik durch Norm}
        \entry{Satz}{Cauchy-Schwarz-Ungleichung}
        \entry{Satz}{Minkowski-Ungleichung}
        
        \entry{Definition}{Metrischer Raum}
        
    \subsection{Cauchy-Folgen}
         
        \entry{Definition}{Cauchy-Folge}
        \entry{Bemerkung}{Konvergenz in Metriken}
        \entry{Satz}{Konvergenz von Cauchy-Folgen}
        
    \subsection{Häufungswerte und Satz von Bolzano-Weierstraß}
    
        \entry{Definition}{Teilfolge}
        \entry{Satz}{Konvergenzkriterien für reelle Zahlenfolgen}
        \entry{Definition}{Häufungswert}
        \entry{Satz}{Satz von Bolzano-Weierstraß}
        \entry{Bemerkung}{Größter und kleinster Häufungswert}
        
    \subsection{Limes Superior und Limes Inferior}
        
        \entry{Definition}{Limes Superior, Limes Inferior}
        \entry{Bemerkung}{Limes Superior, Limes Inferior}[Limes Superior, Limes Inferior - Bemerkung]
        \entry{Satz}{Konvergenzkriterien für beschränkte Folgen}
        \entry{Satz}{Satz von Bolzano-Weierstraß im euklidischen Raum}
        \entry{Satz}{Charakteristische Eigenschaft von Limes Superior und Limes Inferior}
        \entry{Bemerkung}{Ordnungserhaltung bei Grenzwerten}
        \entry{Satz}{Hauptsatz über monotone Folgen}

\section{Unendliche Reihen}

    \subsection{Partialsummen}
    
        \entry{Definition}{Reihe}
        \entry{Notation}{Unendliche Summen}
        \entry{Satz}{Konvergenzkriterium für Reihen}
        \entry{Korollar}{Grenzwert von konvergierenden Reihen}
        \entry{Korollar}{Konvergenzkriterium für absolut konvergente Reihen}
        \entry{Bemerkung}{Absolut konvergent}
        \entry{Satz}{Konvergenz und Beschränktheit}
    
    \subsection{Konvergenzkriterien}
    
        \entry{Satz}{Majorantenkriterium}
        \entry{Satz}{Minorantenkriterium}
        \entry{Satz}{Quotientenkriterium}
        \entry{Satz}{Wurzelkriterium}
        \entry{Bemerkung}{Wurzelkriterium}[Wurzelkriterium - Bemerkung]
        
        \entry{Satz}{Cauchy-Kriterium}
        
        \entry{Satz}{Grenzwertsätze für Folgen positiver reeller Zahlen}
        \entry{Satz}{Cauchyscher Verdichtungssatz}
        
        \entry{Definition}{Alternierende Reihe}
        \entry{Satz}{Leibniz-Kriterium für alternierende Reihen}
        
        \entry{Definition und Satz}{Geometrische Reihe}
        \entry{Definition und Satz}{Harmonische Reihe}
        
    \subsection{Umordnung von Reihen und das Cauchy-Produkt}
    
        \entry{Definition}{Umordnung}
        \entry{Satz}{Umordnungssatz}
        \entry{Definition und Satz}{Cauchy-Produkt}
        
    \subsection{Potenzreihen}
    
        \entry{Definition}{Potenzreihe}
        \entry{Satz}{Konvergenzradius-Kriterium für Potenzreihen}
        \entry{Bemerkung}{Cauchy-Hadamard-Formel}
        \entry{Satz}{Restgliedabschätzung}
        
    \subsection{Exponentialreihe und Eulersche Formel}
    
        \entry{Definition}{Exponentialfunktion}
        \entry{Satz}{Cauchy-Produkt von Potenzreihen}
        \entry{Satz}{Funktionalgleichung der Exponentialfunktion}
        \entry{Satz}{Eulersche Formel}
        \entry{Bemerkung}{Weitere Eigenschaften der Exponentialfunktion}
        \entry{Satz}{Additionstheoreme für Sinus und Cosinus}
        
\section{Grenzwerte und Stetigkeit}

    \subsection{Grenzwerte von Funktionen}
    
        \entry{Definition}{Grenzwert einer Funktion in einem Punkt}
        \entry{Definition}{Einseitige Grenzwerte einer Funktion}
        \entry{Satz}{Existenz des Grenzwerts einer Funktion}
        \entry{Satz}{Folgenkriterium für Grenzwerte von Funktionen}
        \entry{Satz}{Grenzwertrechenregeln für Funktionen}
        \entry{Definition}{Uneigentliche Grenzwerte}
        \entry{Bemerkung}{Vertikale Asymptote}
        \entry{Definition}{Konvergenz auf unendlichen Intervallen}
        \entry{Bemerkung}{Horizontale Asymptote}
        \entry{Definition}{Bestimmt divergent}
        
    \subsection{Stetigkeit von Abbildungen}
    
        \entry{Definition}{Stetige Abbildungen zwischen metrischen Räumen}
        \entry{Definition}{Vektorraum der stetigen Funktionen}
        \entry{Satz}{Epsilon-Delta-Kriterium}
        \entry{Satz}{Alternative Formulierung für Stetigkeit}
        \entry{Bemerkung}{Kurzformulierung von Stetigkeit}
        \entry{Bemerkung}{Unstetig}
        \entry{Definition}{Einseitig stetig}
        \entry{Satz}{Folgenkriterium für Stetigkeit}
        \entry{Notation}{Stetigkeit}[Stetigkeit - Schreibweise]
        \entry{Satz}{Komposition stetiger Abbildungen}
        \entry{Satz}{Stetigkeit von Kompositionen}
        \entry{Bemerkung}{Stabilitätseigenschaft}
        \entry{Satz}{Konsequenz der Stabilitätseigenschaft}
        \entry{Bemerkung}{Sprungstelle, Unstetigkeitsstelle}
        
        \entry{Satz}{Stetigkeit von Potenzreihen}
        
    \subsection{Eigenschaften stetiger reellwertiger Funktionen}
    
        \entry{Satz}{Nullstellensatz}
        \entry{Korollar}{Nullstellensatz}[Nullstellensatz - Anwendung]
        \entry{Satz}{Zwischenwertsatz}
        \entry{Satz}{Fixpunktsatz}
        \entry{Satz}{Minimum und Maximum stetiger Funktionen}
        \entry{Bemerkung}{Bild einer stetigen Funktion auf einem Intervall}
        
    \subsection{Ergänzungen zur Exponentialfunktion und zum Logarithmus}
    
        \entry{Satz}{Stetigkeit des natürlichen Logarithmus}
        \entry{Satz}{Komposition von Logarithmus und Exponentialfunktion}
        
\section{Differentialrechnung}

    \subsection{Differentialrechnung in einer Variablen}
    
        \entry{Definition}{Differenzierbarkeit}
        \entry{Satz}{Stetigkeit aus Differenzierbarkeit}
        \entry{Satz}{Rechenregeln für Ableitungen}
        \entry{Bemerkung}{Potenzregel}
        \entry{Satz}{Kettenregel}
        
        \entry{Bemerkung}{Zweite und weitere Ableitungen}
        
    \subsection{Mittelwertsatz der Differentialrechnung und lokale Extrema}
    
        \entry{Definition}{Lokales Minimum, lokales Maximum}
        \entry{Satz}{Satz von Fermat}
        \entry{Satz}{Satz von Rolle}
        \entry{Satz}{Erster Mittelwertsatz der Differentialrechnung}
        \entry{Korollar}{Monoton fallend, monoton wachsend}
        \entry{Korollar}{Streng monoton fallend, streng monoton wachsend}
        \entry{Satz}{Strenges lokales Minimum, strenges lokales Maximum}
        \entry{Definition}{Stammfunktion}
        \entry{Satz}{Exponentieller Zusammenhang zwischen Ableitung und Funktion}
        \entry{Satz}{Erste Regel von de l'Hospital}
        
        \entry{Satz}{Zweite Regel von de l'Hospital}
        \entry{Satz}{Ableitung der Umkehrfunktion}
        %\entry{Bemerkung}{Herleitung der Ableitung von Umkehrfunktionen}
        %\entry{Beispiel}{Ableitung der Umkehrungen der Trigonometrische Funktionen}
        \entry{Definition}{Hyperbelfunktionen}
        \entry{Bemerkung}{Rechenregeln der Hyperbelfunktionen}