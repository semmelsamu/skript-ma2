Zu jedem $x \in \mathbb{R}$ mit $x > 0$ und $k \in \mathbb{N}$ existiert genau eine reelle Zahl $y > 0$ mit
$$y^k = x.$$
Diese Zahl heißt $k$-te \term{Wurzel} von $x$. Man schreibt $y = \sqrt[k]{x}$ oder auch $y = x^{\frac{1}{k}}$.
Für Wurzeln gelten folgende Rechenregeln:
\begin{itemize}
    \item Für $0 \leq b \leq c$ und $k \in \mathbb{N}$ gilt: $0 < \sqrt[k]{c} - \sqrt[k]{b} \leq \sqrt[k]{c-b}$ (\term{Minkowski-Ungleichung})
\end{itemize}