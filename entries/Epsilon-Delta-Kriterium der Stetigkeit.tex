Eine Funktion $f: D \to \R$ mit $D \in \R$ ist genau dann \term{stetig} an der Stelle $x_0 \in D$, wenn es zu jedem $\varepsilon > 0$ ein $\delta > 0$ gibt, so dass $|f(x) - f(x_0)| < \varepsilon$ für alle $x \in D$ mit $|x - x_0| < \delta$ ist.
$f$ ist also genau dann in $x_0 \in D$ stetig, wenn gilt:
$$\forall \varepsilon > 0 \ \exists \delta > 0 \ \forall x \in D : |x - x_0| < \delta \Longrightarrow |f(x) - f(x_0)| < \varepsilon$$