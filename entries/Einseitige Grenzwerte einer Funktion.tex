\begin{enumerate}[leftmargin=*]
    \item Sei $I \in \mathbb{R}$ ein Intervall, $p$ ein \term{rechter Randpunkt}[Randpunkt] von $I$ und $f$ eine reellwertige Funktion, die auf $I \setminus \{p\}$ (aber eventuell nicht in $p$) definiert ist. Man sagt, dass $f$ bei Annäherung an $p$
    den \term{linksseitigen Grenzwert}[Grenzwert] (oder \term{linksseitigen Limes}[Limes]) $g^-$ besitzt, falls es zu jedem $\varepsilon > 0$ ein (von $\varepsilon$ abhängiges) $\delta > 0$ gibt, sodass gilt:
    $$\text{Für alle $x \in I$ mit $p - \delta < x < p$ gilt} \qquad | f (x) - g^- | < \varepsilon.$$
    \textbf{Notation:}
    $$f(p-) := \lim_{x \to p-} f(x) = g^-$$
    \item Sei $I \in \mathbb{R}$ ein Intervall, $p$ ein \term{linker Randpunkt}[Randpunkt] von $I$ und $f$ eine reellwertige Funktion, die auf $I \setminus \{p\}$ (aber eventuell nicht in $p$) definiert ist. Man sagt, dass $f$ bei Annäherung an $p$
    den \term{rechtsseitigen Grenzwert}[Grenzwert] (oder \term{rechtsseitigen Limes}[Limes]) $g^+$ besitzt, falls es zu jedem $\varepsilon > 0$ ein (von $\varepsilon$ abhängiges) $\delta > 0$ gibt, sodass gilt:
    $$\text{Für alle $x \in I$ mit $p < x < p + \delta$ gilt} \qquad | f (x) - g^+ | < \varepsilon.$$
    \textbf{Notation:}
    $$f(p+) := \lim_{x \to p+} f(x) = g^+$$ 
\end{enumerate}