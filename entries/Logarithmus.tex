Der \term{Logarithmus} ist die Umkehrfunktion der \textit{Exponentialfunktion}. Als Logarithmus einer Zahl $a$ bezeichnet man den Exponenten $x$, mit dem eine Basis $b$ potenziert werden muss, um die Zahl zu erhalten.
$$b^x = a \qquad \Longleftrightarrow \qquad x = \log_b a$$
Insbesondere gelten folgende Zusammenhänge:
$$\log_b b = 1, \qquad \log_b 1 = 0$$
Für Logarithmen mit Basis $10$ oder Basis $e$ gibt es spezielle Schreibweisen, und zwar:
\begin{description}
    \item[\term{Dekadischer Logarithmus}] $\log a := \log_{10} a$.
    \item[\term{Natürlicher Logarithmus}] $\ln a := \log_e a$. Dabei ist $e \approx 2,71828\dots$ die \term{eulersche Zahl}[Eulersche Zahl].
    \item[\term{Binärer Logarithmus}] $\text{lb} \ a := \log_2 a$.
\end{description}
Für Logarithmen gelten folgende Rechenregeln:
\begin{itemize}
    \item $\ln(1) = 0$
    \item $1-\frac{1}{x} \leq \ln(x) \leq x-1$ und \\
    $2(1-\frac{1}{\sqrt{x}}) \leq \ln(x) \leq 2 (\sqrt{x} - 1)$
    \item $\ln(xy) = \ln(x) + \ln(y)$ (\term{Funktionalgleichung})
    \item $\ln(\frac{1}{x}) = -\ln(x)$
    \item $0<x<y \Longrightarrow \ln(x) < \ln(y)$
\end{itemize}