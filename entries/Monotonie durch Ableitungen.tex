Sei $f: [a, b] \to \R$ differenzierbar in $(a, b)$. Dann gilt: $f$ ist
\begin{description}
    \item[\term{Monoton fallend},] falls $f'(x) \leq 0 \ \forall x \in (a, b)$,
    \item[\term{Monoton steigend},] falls $f'(x) \geq 0 \ \forall x \in (a, b)$,
    \item[\term{Streng monoton fallend},] falls $f'(x) < 0 \ \forall x \in (a, b)$,
    \item[\term{Streng monoton steigend},] falls $f'(x) > 0 \ \forall x \in (a, b)$,
    \item[\term{Konstant},] falls $f'(x) = 0 \ \forall x \in (a, b)$.
\end{description}
Für monoton fallend, monoton steigend und konstant gelten die Implikationen auch in die andere Richtung.