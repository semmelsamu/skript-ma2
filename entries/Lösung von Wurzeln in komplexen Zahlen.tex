Für $n \in \mathbb{N}$ und $\xi \in \mathbb{C} \backslash \{0\}$ hat die Gleichung 
$$z^n = \xi \ [= (\varphi, \psi)] = \varphi \cdot (\cos(\psi)+ i \sin(\psi))$$
für gegebenes $\xi$ genau $n$ verschiedene Lösungen (Wurzeln):
$$z_j := \sqrt[n]{\varphi} \left[ \cos \left( \frac{\psi+2\pi j}{n} \right) + i \sin \left( \frac{\psi+2\pi j}{n} \right) \right]$$
für $0 \leq j \leq n - 1$. Diese bilden die Ecken eines regelmäßigen $n$-Ecks auf dem Kreis mit dem Mittelpunkt $(0, 0)$ und Radius $|z| = \sqrt[n]{\varphi}$.
Im Spezialfall $\xi = 1 = 1+0i$ erhält man als Lösungen der Gleichung 
$$z^n = 1$$
die $n$-ten Einheitswurzeln
$$z_j := \cos \left( \frac{2\pi j}{n} \right) + i \sin \left( \frac{2\pi j}{n} \right)$$
mit $0 \leq j \leq n - 1$.