Sei $X$ ein $K$-Vektorraum ($K = \mathbb{R}$ oder $\mathbb{C}$). Eine Abbildung
$$||\cdot||: X \to \mathbb{R}$$
heißt \term{Norm}, wenn für alle $x, y \in X$ gilt:
\begin{enumerate}[label="",leftmargin=0pt]
    \item \term{Positiv definit} \, $||x|| \geq 0$, wobei $||x|| = 0$ genau für $x = 0$.
    \item \term{Homogen} \, $||\lambda x|| = |\lambda| \cdot ||x||$.
    \item \term{Dreiecksungleichung} \, $||x+y|| \leq ||x|| + ||y||$.
\end{enumerate}