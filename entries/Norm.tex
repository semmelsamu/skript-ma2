Sei $X$ ein $K$-Vektorraum ($K = \mathbb{R}$ oder $\mathbb{C}$). Eine Abbildung
$$||\cdot||: X \to \mathbb{R}$$
heißt \term{Norm}, wenn für alle $x, y \in X$ gilt:
\begin{enumerate}[label=(N\arabic*)]
    \item $||x|| \geq 0$ und $||x|| = 0$ $\Leftrightarrow$ $x = 0$ (Nullvektor)
    \item $||\lambda x|| = |\lambda| \cdot ||x||$
    \item $||x+y|| \leq ||x|| + ||y||$ (Dreiecksungleichung)
\end{enumerate}