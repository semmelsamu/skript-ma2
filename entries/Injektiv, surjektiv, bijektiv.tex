Eine Abbildung von einer Definitionsmenge $D$ auf eine Zielmenge $Z$ mit $f \mapsto f(x)$ heißt
\begin{description}
    \item \term{injektiv}, falls für $x, x' \in D$ aus $x \neq x'$ stets auch $f(x) \neq f(x')$ folgt. \\
    Einfach gesagt, wenn jedem $f(x)$ \textit{höchstens ein} $x$ zugeordnet wird.
    \item \term{surjektiv}, falls für jedes $y \in Z$ ein $x \in D$ mit $f(x) = y$ existiert. \\
    Einfach gesagt, wenn jedem $f(x)$ \textit{mindestens ein} $x$ zugeordnet wird.
    \item \term{bijektiv}, falls sie injektiv und surjektiv ist. \\
    Einfach gesagt, wenn jedem $f(x)$ \textit{genau ein} $x$ und jedem $x$ \textit{genau ein} $f(x)$ zugeordnet wird.
\end{description}