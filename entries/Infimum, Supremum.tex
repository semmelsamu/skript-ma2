\begin{enumerate}[label=(\alph*)]
    \item Eine Zahl $s \in \mathbb{R}$ heißt \term{Infimum} von $A$, falls gilt:
        \begin{enumerate}[label=\arabic*.]
            \item $s$ ist untere Schranke von $A$
            \item Für jede untere Schranke $t$ von $A$ gilt $t \leq s$.
        \end{enumerate}
        Bedeutet: $s$ ist die \term{größte}[] untere Schranke von $A$.
    \item Entsprechend definiert man eine Zahl $s \in \mathbb{R}$ als \term{Supremum} einer Menge $A \in \mathbb{R}$, falls gilt:
        \begin{enumerate}[label=\arabic*.]
            \item $s$ ist obere Schranke von $A$
            \item Für jede obere Schranke $t$ von $A$ gilt $t \geq s$.
        \end{enumerate}
        Bedeutet: $s$ ist die \term{kleinste}[] obere Schranke von $A$
\end{enumerate}