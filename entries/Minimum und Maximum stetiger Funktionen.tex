Sei $f: [a, b] \to \mathbb{R}$ stetig. Dann gilt:
\begin{enumerate}
    \item $f$ ist beschränkt, also existieren in $\mathbb{R}$ 
    $$m := \inf(\underbrace{f([a, b])}_{\mathclap{:= \{f(x) \ | \ x \in [a, b]\} = \text{Bild}(f)}}) \quad \text{und} \qquad M := \sup(f([a, b])).$$
    \item $m$ und $M$ werden als Funktionswerte angenommen, d.h. es gilt $x_m, x_M \in [a, b]$ mit
    $$f(x_m) = m \qquad \text{und} \qquad f(x_M) = M.$$
\end{enumerate}
Man nennt $m$ bzw. $M$ das Minimum bzw. Maximum fon $f$ sowie $x_m$ bzw. $x_M$ Minimum- bzw. Maximumstelle.