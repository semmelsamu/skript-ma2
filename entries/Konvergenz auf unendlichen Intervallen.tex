\begin{enumerate}[leftmargin=*]
    \item Sei $I = [p, \infty)$, $f: I \to \R$ eine Funktion und $g \in \R$. Man sagt: $f(x)$ strebt für $x$ gegen $+\infty$ gegen $g$, falls gilt:
    $$\text{Zu jedem $\varepsilon > 0$ git es ein $S > 0$, sodass $|f(x)-g| < \varepsilon$ für $x > S$ ist.}$$
    \textbf{Notation:} $$\lim_{x \to \infty} f(x) = g$$
    \item Sei $I = (-\infty, p]$, $f: I \to \R$ eine Funktion und $g \in \R$. Man sagt: $f(x)$ strebt für $x$ gegen $-\infty$ gegen $g$, falls gilt:
    $$\text{Zu jedem $\varepsilon > 0$ git es ein $S > 0$, sodass $|f(x)-g| < \varepsilon$ für $x < -S$ ist.}$$
    \textbf{Notation:} $$\lim_{x \to -\infty} f(x) = g$$
\end{enumerate}