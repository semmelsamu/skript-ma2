Seien $f, g: I \to \R$ im Punkt $x \in I$ differenzierbar und $c \in \R$. Dann sind auch $c \cdot f$, $f+g$ und $f \cdot g$ differenzierbar. Ist $g(x) \neq 0$, so ist auch $\frac{f}{g}$ differenzierbar in $x$, und es gilt:
\begin{enumerate}[leftmargin=0pt,label=""]
    \item \term{Vorfaktor} \, $(c \cdot f)' = c \cdot f'$
    \item \term{Summenregel} \, $(f \pm g)' = f' \pm g'$
    \item \term{Produktregel} \, $(f \cdot g)' = f' \cdot g + f \cdot g'$
    \item \term{Quotientenregel} \, $\left(\frac{f}{g}\right)' = \frac{f' g - f g'}{g^2}$
\end{enumerate}