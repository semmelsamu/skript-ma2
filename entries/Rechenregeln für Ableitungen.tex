Seien $f, g: I \to \R$ im Punkt $x \in I$ differenzierbar. Dann sind auch $c \cdot f$ (mit $c \in \R$), $f+g$ und $f \cdot g$ differenzierbar. Ist $g(x) \neq 0$, so ist auch $\frac{f}{g}$ differenzierbar. Es gilt: \par
\vspace{0.5em}
\begin{tabular}{@{}ll@{}}
    \term{Vorfaktor} \, $(c \cdot f)' = c \cdot f'$ \hspace{2.5cm} &
    \term{Summenregel} \, $(f + g)' = f' + g'$ \vspace{0.8em} \\
    \term{Produktregel} \, $(f \cdot g)' = f' \cdot g + f \cdot g'$ &
    \term{Quotientenregel} \, $\left(\dfrac{f}{g}\right)' = \dfrac{f' g - f g'}{g^2}$
\end{tabular}