Seien $\sum a_n$ und $\sum b_n$ zwei \textit{konvergente} Reihen. Für $n \in \mathbb{N}_0$ setze man
$$c_n := \sum_{k=0}^n a_k b_{n-k} = a_0 b_n + a_1 b_{n-1} + \dots + a_n b_0.$$
Wenn $\sum a_n$ \textit{absolut} konvergiert, dann ist die Reihe $\sum c_n$ konvergent mit
$$\sum_{n=0}^\infty c_n = \sum_{n=0}^\infty \left( \sum_{n=0}^\infty a_k b_{n-k} \right) = \left( \sum_{n=0}^\infty a_n \right) \left(\sum_{n=0}^\infty b_n \right)$$
$\sum c_n$ heißt das \term{Cauchy-Produkt} der Reihen $\sum a_n$ und $\sum b_n$. \par
\textbf{Achtung:} Wenn beide Reihen nicht absolut konvergent sind, dann ist deren Cauchy-Produkt nicht immer konvergent.