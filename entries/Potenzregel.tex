\begin{enumerate}[leftmargin=*]
    \item Damit sind alle Polynome differenzierbar und alle rationalen Funktionen $\frac{p}{q}$ außer an den Stellen, in denen $q$ Null wird.
    \item Mit der Quotientenregel erhält man für $f: \R \setminus \{0\} \to \R, f(x) = \frac{1}{x^n} (n \in \N)$
    $$f'(x) = \frac{0 \cdot x^n - 1 \cdot n \cdot x^{n-1}}{x^{2n}} = \frac{-n}{x^{n+1}}$$
\end{enumerate}
Insgesamt ergibt sich die \term{Potenzregel}:
$$(x^n)' = n \cdot x^{n-1} \qquad \forall n \in \mathbb{Z}$$