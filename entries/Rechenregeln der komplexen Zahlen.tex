Seien $z, w \in \mathbb{C}$. Dann gilt: 
\begin{description}
    \item[\term{Konjugation der Summe}] $\overline{z \pm w} = \overline{z} \pm \overline{w}$
    \item[\term{Konjugation der Konjugation}] $\overline{(\overline{z})} = z$
    \item[\term{Realteil und Imaginärteil}] $\Re(z) = \frac{1}{2}(z+\overline{z})$,
    $\Im(z) = \frac{1}{2i}(z-\overline{z})$
    \item[\term{Absolutwert des Realteils und Imaginärteils}] $|\Re(z)| \leq |z|$ und
    $|\Im(z)| \leq |z|$
    \item[\term{Abschätzung des Absolutwerts}] $|z| \leq |\Re(z)| + |\Im(z)|$
    \item[\term{Betrag}] $|z| = \sqrt{z \cdot \overline{z}} = \sqrt{a^2+b^2}$,
    $z = (a, b)$
    \item[\term{Gleichheit des Absolutwerts}] $|z| = |\overline{z}| = |-z|$
    \item[\term{Absolutwert des Produkts}] $|z \cdot w| = |z| \cdot |w|$
    \item[\term{Nichtnegativität des Absolutwerts}] $|z| \geq 0 \land |z| = 0 \Leftrightarrow z = 0$
    \item[\term{Multiplikation mit der konjugierten Komplexen}] $z \cdot \overline{z} = (|z|)^2$
    \item[\term{Dreiecksungleichung}] $|z+w| \leq |z| + |w|$ und $\big||z|-|w|\big| \leq |z-w|$
\end{description}