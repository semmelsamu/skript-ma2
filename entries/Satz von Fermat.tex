Sei $f: [a, b] \to \R$ eine Funktion, die in einem \textit{inneren} Punkt $p$ ein lokales Extremum besitzt, so gilt (sofern $f'(x)$ existiert):
$$f'(p) = 0.$$

\textbf{Anders ausgedrückt:} Die Ableitung einer differenzierbaren Funktion hat an lokalen Extrema eine Nullstelle.

\textbf{Achtung:} Die Umkehrung (also dass jede Nullstelle ein lokales Extremum impliziert) gilt nicht.