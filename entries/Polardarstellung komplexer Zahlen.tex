Neben den kartesischen Koordinaten benutzt man zur Beschreibung komplexer Zahlen häufig \term{Polarkoordinaten} $(r, \varphi)$.
Ist $z = (x, y) \neq (0, 0) \in \mathbb{C}$, so existiert $(r, \varphi) \in (0, \infty) \times [0, 2 \pi)$ mit
$$x = r \cdot \cos(\varphi) \quad \text{und} \quad y = r \cdot \sin(\varphi)$$
wobei $r := \sqrt{x^2+y^2} = |z|$. $\varphi$ wird im Bogenmaß angegeben. Man erhält damit die Polarkoordinatendarstellung
$$z = r(\cos(\varphi) + i \sin(\varphi)) = re^{i\varphi}.$$
Für $0 \leq \varphi < 2 \pi$ setzt man $\varphi = \text{arg}(z)$ und nennt $\varphi$ das \term{Argument} von z.