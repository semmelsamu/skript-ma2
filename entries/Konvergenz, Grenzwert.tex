Sei eine $(a_n)$ eine Folge. Die Folge \term{konvergiert} gegen $a \in \mathbb{R}$ (bzw. $\mathbb{C}$), falls gilt:
Zu jedem $\varepsilon > 0$ existiert ein $N = N(\varepsilon)$ ("$N$ hängt von $\varepsilon$ ab"), $N \in \mathbb{N}$ mit folgender Eigenschaft:
$$|a_n - a| < \varepsilon \qquad \text{für alle $n \geq N(\varepsilon)$}$$
Die Zahl $a$ heißt dann \term{Grenzwert} der Folge (\term{Limes}).