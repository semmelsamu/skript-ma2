Durch die injektive Abbildung (\term{Einbettung})
\begin{align*}
    e : \mathbb{R} & \longrightarrow \mathbb{R} \times \mathbb{R} = \mathbb{C} \\
    x & \longmapsto e(x) := (x, 0)
\end{align*}

lässt sich eine reelle Zahl $x$ mit der komplexen Zahl $(x, 0)$ identifizieren, und somit lässt sich $\mathbb{R}$ als Teilmenge von $\mathbb{C}$ auffassen.