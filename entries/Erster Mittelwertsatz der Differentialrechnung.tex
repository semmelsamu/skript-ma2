Sei $f: [a, b] \to \R$ stetig und in $(a, b)$ differenzierbar. Dann existiert mindestens ein $x_0 \in (a, b)$ mit
$$f'(x_0) = \frac{f(b)-f(a)}{b-a}.$$
Geometrisch gedeutet bedeutet dies, dass die Sekantensteigung an mindestens einer Stelle zwischen $a$ und $b$ als Steigung der Tangente am Funktionsgraph auftritt. 