\begin{enumerate}[leftmargin=*]
    \item Sei $I = [p, +\infty)$, $f: I \to \R$ eine Funktion. Man sagt: $f(x)$ besitzt für $x$ gegen $+\infty$ den \term{uneigentlichen Grenzwert} $+\infty$ (bzw. $f$ sei für $x$ gegen $+\infty$ \term{bestimmt divergent gegen} $+\infty$), falls gilt: $$\text{Zu jedem $S > 0$ gibt es ein $K = K(S) \in \R$, sodass $f(x) > S$ für $x > K$ ist.}$$ \textbf{Notation:} $$\lim_{x \to -\infty} f(x) = +\infty$$
    \item Sei $I = [p, +\infty)$, $f: I \to \R$ eine Funktion. Man sagt: $f(x)$ besitzt für $x$ gegen $+\infty$ den \term{uneigentlichen Grenzwert} $-\infty$ (bzw. $f$ sei für $x$ gegen $+\infty$ \term{bestimmt divergent gegen} $-\infty$), falls gilt: $$\text{Zu jedem $S > 0$ gibt es ein $K = K(S) \in \R$, sodass $f(x) < -S$ für $x > K$ ist.}$$ \textbf{Notation:} $$\lim_{x \to +\infty} f(x) = -\infty$$
    \item Sei $I = (-\infty, p]$, $f: I \to \R$ eine Funktion. Man sagt: $f(x)$ besitzt für $x$ gegen $-\infty$ den \term{uneigentlichen Grenzwert} $+\infty$ (bzw. $f$ sei für $x$ gegen $-\infty$ \term{bestimmt divergent gegen} $+\infty$), falls gilt: $$\text{Zu jedem $S > 0$ gibt es ein $K = K(S) \in \R$, sodass $f(x) > S$ für $x < K$ ist.}$$ \textbf{Notation:} $$\lim_{x \to -\infty} f(x) = +\infty$$
    \item Sei $I = (-\infty, p]$, $f: I \to \R$ eine Funktion. Man sagt: $f(x)$ besitzt für $x$ gegen $-\infty$ den \term{uneigentlichen Grenzwert} $-\infty$ (bzw. $f$ sei für $x$ gegen $-\infty$ \term{bestimmt divergent gegen} $-\infty$), falls gilt: $$\text{Zu jedem $S > 0$ gibt es ein $K = K(S) \in \R$, sodass $f(x) < -S$ für $x < K$ ist.}$$ \textbf{Notation:} $$\lim_{x \to -\infty} f(x) = -\infty$$
\end{enumerate}