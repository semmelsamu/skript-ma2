Seien $a_n := (\frac{n+1}{n})^n$, $b_n := (\frac{n+1}{n})^{n+1}$ und $I_n := [a_n, b_n]$ mit $n \in \mathbb{N}$. Weiter ist $(I_n)_{n \in \mathbb{N}}$ eine Intervallschachtelung. Dann bezeichnet man
$$\bigcap_{n \in \mathbb{N}} I_n := I_1 \cap I_2 \cap \dots$$
als \term{eulersche Zahl} $e$ ($\approx 2,71828 \dots$).