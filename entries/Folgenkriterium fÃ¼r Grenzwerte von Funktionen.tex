Sei $I \in \mathbb{R}$ ein Intervall, $p \in I$ ein innerer Punkt oder ein Randpunkt von $I$ und $f$ eine reellwertige Funktion, die auf $I \setminus {p}$ (aber eventuell nicht in $p$) definiert ist. Dann sind folgende Aussagen äquivalent:
\begin{enumerate}
    \item Der Grenzwert $g := \lim\limits{x \to p} f(x)$ existiert.
    \item Für \textit{jede} Folge $(x_n)_{n \in \mathbb{N}} \in I$ mit $x_n \neq p$ für alle $n \in \mathbb{N}$ und $\lim\limits_{n \to \infty} x_n = p$ ist $\lim\limits_{n \to \infty} f(x_n) = g$.
\end{enumerate}