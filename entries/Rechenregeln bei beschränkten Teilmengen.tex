Seien $A$ und $B$ nichtleere und nach oben beschränkte Teilmengen von $\mathbb{R}$.

\begin{itemize}

    \item Ist $A \subseteq B$, so gilt 
    $$\sup A \leq \sup B$$
    
    \item Ist $A+B := \{a + b \ | \ a \in A, b \in B\}$, so ist
    $$\sup(A+B) = \sup A + \sup B$$
    
    \item Ist $\lambda A := \{\lambda \cdot a \ | \ a \in A\}$ (mit $\lambda \in \mathbb{R})$, so gilt
    $$\sup(\lambda A) = \lambda \sup A \quad \text{(mit $\lambda > 0$)}$$
    
    \item Ist $A \cdot B := \{a \cdot b \ | \ a \in A, b \in B\}$, so gilt
    $$\sup(A \cdot B) = \sup A \cdot \sup B \quad \text{falls $A, B \subseteq [0, \infty[$}$$
    
\end{itemize}