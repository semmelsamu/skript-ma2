Sei $K$ ein angeordneter Körper und $a, b, c \in K$. Dann gilt:
\begin{enumerate}[label="",leftmargin=0pt]
    \item \term{Transitivität} \, $a < b \land b < c \Longrightarrow a < c$
    \item \term{Additionseigenschaft} \, $a < b \Longleftrightarrow a + c < b + c$
    \item \term{Skalarmultiplikation} \, $a < b \land c > 0 \Longrightarrow ca < cb$
    \item \term{Inversion} \, $a < b < 0 \Longrightarrow \frac{1}{a} < \frac{1}{b}$
    \item \term{Monoton} \, $0 \leq a \leq b \Longrightarrow a^2 < b^2$
    \item \term{Positivität des Quadrats} \, $\forall a \neq 0 : a^2 > 0$, insbesondere $1^2 = 1 > 0$
\end{enumerate}