Gegeben sei eine reelle Zahlenfolge $(a_n)_{n \in \N}$.
\begin{enumerate}
    \item Ist die Folge \term{nach oben beschränkt}, dann existiert zu jedem $\varepsilon > 0$ eine Zahl $n_0 \in \N$, sodass
    $$a_n < (\limsup_{n \to \infty} a_n) + \varepsilon \qquad \text{für alle $n \geq n_0$.}$$
    \item Ist die Folge \term{nach unten beschränkt}, dann existiert zu jedem $\varepsilon > 0$ eine Zahl $n_0 \in \N$, sodass
    $$a_n > (\liminf_{n \to \infty} a_n) - \varepsilon \qquad \text{für alle $n \geq n_0$.}$$
    \item Ist die Folge \term{beschränkt}, dann existiert zu jedem $\varepsilon > 0$ eine Zahl $n_0 \in \N$, sodass
    $$(\liminf_{n \to \infty} a_n) - \varepsilon < a_n < (\limsup_{n \to \infty} a_n) + \varepsilon \qquad \text{für alle $n \geq n_0$.}$$
\end{enumerate}