Seien $(X, d_X), (Y, d_Y), (Z, d_Z)$ drei metrische Räume und $p \in X$. Falls $f: X \to Y$ stetig in $p$ und $g: Y \to Z$ stetig in $f(p)$ ist, so ist $g \circ f: X \to Z$ stetig in $p$.

Ist $X$ ein metrischer Raum, $f: X \to \mathbb{K}$, $g: X \to \mathbb{K}$ und $\lambda \in \mathbb{K}$, so sind durch
$$(f+g)(x) := f(x) + g(x), \qquad (\lambda \cdot f)(x) := \lambda \cdot f(x), \qquad (f \cdot g)(x) := f(x) \cdot g(x)$$
weitere Abbildungen $f+g$, $\lambda \cdot f$ und $f \cdot g: X \to \mathbb{K}$ erklärt.

Ist darüber hinaus $f(x) \neq 0$ für alle $x \in X$, so ist $\frac{1}{f} : X \to \mathbb{K}$ definiert durch $\left(\frac{1}{f}\right)(x) := \frac{1}{f(x)}$.