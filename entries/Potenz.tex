Eine \term{Potenz} ist eine verkürzte Schreibweise für das \textit{Multiplizieren} einer \term{Basis} $b$ mit sich selbst, und zwar so oft, wie im \term{Exponenten}[Exponent] $x$ steht.
$$b^x := \underbrace{b \cdot b \cdot \dots \cdot b}_\text{$x$ mal}$$
Insbesondere gelten folgende Zusammenhänge:
$$a^0 = 1, \ \text{insbesondere ist $0^0 = 1$}, \qquad a^1 = a$$
Spezielle Potenzen können auch folgendermaßen notiert werden:
$$a^{-n} := \frac{1}{a^n}, \qquad a^{\frac{1}{n}} := \sqrt[n]{a}$$
Außerdem gelten folgende Rechenregeln:
\begin{itemize}
    \item $a^n \cdot b^n = (a \cdot b)^n, \quad a^n : b^n = (a : b)^n$
    \item $a^n \cdot a^m = a^{n+m}, \quad a^n : a^m = a^{n-m}$
    \item $(a^n)^m = a^{n \cdot m}$
    \item $a \cdot b^n \neq (a \cdot b)^n$
\end{itemize}