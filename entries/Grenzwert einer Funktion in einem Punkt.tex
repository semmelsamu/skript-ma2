Sei $I \in \mathbb{R}$ ein Intervall, $p \in I$ ein \term{innerer Punkt} und $f$ eine reellwertige Funktion, die auf $I \setminus \{p\}$ (aber eventuell nicht in $p$) definiert ist. Man sagt, dass $f$ bei Annäherung an $p$ den (beidseitigen) \term{Grenzwert} (oder \term{Limes}) $g$ besitzt, falls es zu jedem $\varepsilon > 0$ ein (von $\varepsilon$ abhängiges) $\delta > 0$ gibt, sodass gilt:
$$\text{Für alle $x \in I$ mit $0 < | x - p | < \delta$ gilt} \qquad | f (x) - g | < \varepsilon.$$
\textbf{Notation:}
$$\lim_{x \to p} f(x) = g$$