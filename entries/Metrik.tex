Sei $M \neq \emptyset$ eine Menge. Eine Abbildung 
$$d: M \times M \to \mathbb{R}$$
heißt \term{Metrik}, wenn für alle $a, b, c \in M$ gilt:
\begin{description}
    \item[\term{Positiv definit}] $d(a, b) \geq 0$, wobei $d(a, b) = 0$ genau für $a=b$.
    \item[\term{Symmetrisch}] $d(a, b) = d(b, a)$.
    \item[\term{Dreiecksungleichung}] $d(a, c) \leq d(a, b) + d(b, c)$.
\end{description}
$(M, d)$ heißt \term{metrischer Raum} und $d(a, b)$ der \term{Abstand} zwischen $a$ und $b$.