\begin{enumerate}[leftmargin=*]
    \item $\exp(1) = \sum\limits_{k=0}^\infty \frac{1}{k!} = e = \lim\limits_{n \to \infty} (1 + \frac{1}{n})^n$.
    \item $\forall z \in \mathbb{C}$ gilt $\exp(-z) = \frac{1}{\exp(z)}$, und insbesondere ist $\exp(z) \neq 0$, \textit{denn} $\exp(z) \cdot \exp(-z) = \exp(0) = 1$.
    \item $\forall x \in \mathbb{R}$ ist $\exp(x) > 0$. Genauer gilt $\exp(x) > 1$ für $x > 0$ und $0 < \exp(x) < 1$ für $x < 0$.
    \item $\forall x, y \in \mathbb{R}: x<y \Rightarrow \exp(x) < \exp(y)$.
    \item $\forall t \in \mathbb{R}: |\exp(i \cdot t)| = 1$.
\end{enumerate}