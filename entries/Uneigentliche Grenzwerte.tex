\begin{enumerate}[leftmargin=*]
    \item Sei $I \in \mathbb{R}$ ein Intervall, $p$ ein \term{rechter Randpunkt}[Randpunkt] von $I$ und $f$ eine reellwertige Funktion, die aber \textit{nicht} in $p$ definiert ist. Man sagt, dass $f$ bei Annäherung an $p$ den \term{linksseitigen uneigentlichen Grenzwert}[uneigentlicher Grenzwert] $+\infty$ (bzw. $-\infty$) besitzt, falls es zu jedem $S > 0$ ein $\delta > 0$ gibt, sodass gilt:
    $$\text{Für alle $x \in I$ mit $p - \delta  < x < p$ gilt} \qquad f(x) > S \text{ (bzw. $f(x) < -S$).}$$
    \textbf{Notation:} $$\lim_{x \to p-} f(x) = \infty \qquad \text{(bzw. $-\infty$)}$$
    \item Sei $I \in \mathbb{R}$ ein Intervall, $p$ ein \term{linker Randpunkt}[Randpunkt] von $I$ und $f$ eine reellwertige Funktion, die aber \textit{nicht} in $p$ definiert ist. Man sagt, dass $f$ bei Annäherung an $p$ den \term{rechtsseitigen uneigentlichen Grenzwert}[uneigentlicher Grenzwert] $+\infty$ (bzw. $-\infty$) besitzt, falls es zu jedem $S > 0$ ein $\delta > 0$ gibt, sodass gilt:
    $$\text{Für alle $x \in I$ mit $p < x < p + \delta$ gilt} \qquad f(x) > S \text{ (bzw. $f(x) < -S$).}$$
    \textbf{Notation:} $$\lim_{x \to p+} f(x) = \infty \qquad \text{(bzw. $-\infty$)}$$
\end{enumerate}