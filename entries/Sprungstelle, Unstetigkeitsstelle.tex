Ist eine Funktion $f$ in einer Umgebung eines Punktes $p$ definiert, in $p$ selbst aber nicht stetig, so nennt man $p$ eine \term{Unstetigkeitsstelle} von $f$ bzw. sagt, dass $f$ in $p$ unstetig ist. Es gibt dann zwei Möglichkeiten:
\begin{enumerate}
    \item Wenn $f(p-)$ und $f(p+)$ \textit{beide} existieren, aber \textit{nicht} gleich sind, dann nennt man $p$ eine \term{Sprungstelle} (Unstetigkeitsstelle 1. Art, einfache Unstetigkeitsstelle) von $f$ und den Wert $|f(p+) - f(p-)|$ die \term{Sprunghöhe}.
    \item Zumindest einer der beiden einseitigen Grenzwerte existiert \textit{nicht}. Dann nennt man $p$ eine \term{wesentliche Unstetigkeitsstelle} (Unstetigkeitsstelle 2. Art) von $f$.
\end{enumerate}



